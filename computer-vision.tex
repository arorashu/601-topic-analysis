\documentclass{article}
\usepackage{hyperref}
\usepackage{graphicx}
\usepackage{float}
\begin{document}
	\begin{titlepage}
		\vspace*{\stretch{1.0}}
		\begin{center}
			\Large\textbf{Deep Learning for Computer Vision - A crash course}\\
			\large\textit{Shubham Arora}
		\end{center}
		\vspace*{\stretch{2.0}}
	\end{titlepage}

	\tableofcontents
	\newpage
	
	\section{Introduction}
		\paragraph{Computer Vision}
		A subfield of Artificial Intelligence and Machine Learning that deals with tasks that require high level understanding of images and videos. The expected benchmark in any such task is to reach human level accuracy. Human level accuracy is anywhere between 2-5\% ()depending on task)
		\paragraph{Deep Learning}
		Artificial Neural Networks that are characterized by having multiple layers.The goal of a neural networks is to approximate some function f().
		Intuitively, we want the Neural Networks to learn increasing level of abstraction in successive layers.
		\begin{figure}[H]
			\includegraphics[width=10cm]{NN-dlbook.jpeg}
			\caption{Example of a NN performing image classification}
			\label{fig:NN-dlbook.jpeg}
		\end{figure}
		
	
	\section{Standard tasks in Computer Vision}
	\begin{itemize}
		\item Recognition
		\begin{itemize}
			\item Image classification
			\item Image captioning
			\item Object localization
			\item Segmentation
			\item Object Detection
		\end{itemize}
		\item Motion Analysis
		\begin{itemize}
			\item Tracking
			\item Optical flow
		\end{itemize}
		\item Other
		\begin{itemize}
			\item Image recolorization
			\item Super-Resolution
		\end{itemize}
	
	\end{itemize}
	\section{Important Ideas of Deep Learning-Vision. A historical perpective}
	
	\subsection{Convolutional Neural Networks (CNN)}
		\paragraph{LeNet}
			1998 - LeNet-5 was one of the first NN that utilized backpropagation using Supervised Learning. CNN's are by far the most used Neural layer architecture in any machine vision tasks. Yann LeCun was a co-recipent of the 2018 Turing Award for his work in Ai \& vision, owning a lot to the success of CNN in practical applications.
			
		\paragraph{How does it work}
		\begin{itemize}
			\item ConvNet architecture is particularly optimized for images - vastly reduces the number of parameters neede to train the network.
			\item essentially a combination of matrix dot products and $max()$ operations 
			\item Function from raw pixels $->$ n numbers (classs scores in the case of image classification) 
			
		\end{itemize}
			
	\subsection{Deep Convolutional Neural Networks (CNN)}
		\paragraph{AlexNet}
		2012 - Image classifier on the ImageNet database. It Built upon the 1998 work on CNN, it was scaled massively due to availability of exponentially more data and parallel compute using GPU's
		\paragraph{Major Improvements}
		\begin{itemize}
			\item Reduced code complexity as it utilized homogeneous architectures
			\item No need to do complex step of feature extraction
			\item Made it easier to do Transfer Learning
		\end{itemize}
		\paragraph{Industry applications}
		\begin{itemize}
			\item Face Recognition
			\item Self driving cars
			\item Image captioning?
			\item Building block in Reinforcement Learning
		\end{itemize} 
	
	\subsection{Residual Networks (ResNet)}
		\paragraph{ResNet}
		2015 - Microsoft Research - Much better performance than plain deep neural networks. Utilized the idea of skip connections across non sequential layers. Won the ImageNet challenge in 2015.

	\section{Why vision is important?}
	\begin{itemize}
		\item Vision and NLP have been the core of Machine Learning innovation
		\item Vision is a very important ability in robotics for perception, localization, mapping and motion planning tasks.
		\item Automate critical tasks that rely on human vision - fault detection, autonomous vehicles
	\end{itemize}

	
	\section{Important Applications}
		\subsection{Object Detection}
		It is the task of assiging a label to an image from a fixed set of categories.
		\begin{figure}[H]
			\includegraphics[width=10cm]{classify-cat.png}
			\caption{Example of classifying a cat image}
			\label{fig:classify-cat}
		\end{figure}
		As shown in Figure:\ref{fig:classify-cat}, thse system outputs various probabilities for different possible categories.
		
		\subsection{Human Activity Recognition}
		The task of identifying specific movement or action using data from visual sensors. These activities are like: walking, jumping, talking, standing, etc. The task can be online or offline depending upon the real time nature of the application. With the dawn of cheap consumer electronics like smartphones and cameras, lots of data is available now, which made Deep Learning viable and very effective.


%\section{Areas of active research}
%\paragraph{Transfer Learning}	

	\section{Important Tools}
	\begin{itemize}
		\item OpenCV
		\item Keras
		\item Tensorflow
	\end{itemize}
	
	
	\section{Peer Review}
		\subsection{Thuy Pham}
			\paragraph{Topics covered}
			Thuy analyses Unsupervised learning approaches in practice and the field of machine ethics. She discusses:
			\begin{itemize}
				\item Unsupervised learning can improve performance when used alongwith the traditional supervised learning approaches.
				\item Fraud detection in credit card and auto insurance industries
				\item The emerging field of Machine Ethics.She outlines the philosophy, need and challenges in codifying ethics in AI systems.
			\end{itemize}  
			\paragraph{Learnings}
			I learned about how clustering and supervised learning are used together in detecting credit card anomaly,
			the method of Spectral Ranking Anomaly for auto insurance fraud detection and challenges in coding ethics into machines.

		\subsection{Bowen Qin}
			\paragraph{Topics covered}
			Bowen does a deep analysis of the TensorFlow platform for use in creating Machine Learning models using Neural Networks. She discusses the following:
			\begin{itemize}
				\item How Tensorflow has tooling and support for multiple programming languages, neural network arcchitectures and deployment environments
				\item An analysis of common tasks performed using TF - Image classification and segmentation using CNN's
				\item How TF enables quick learning, prototyping and development
			\end{itemize}
			\paragraph{Learnings}
			I learned how Tensorflow is much more than an ML library. The community and tooling support for TF alongwith backing by Google makes it a very lucrative ML tool.
			
		\subsection{Mengting Song}
			\paragraph{Topics covered}
			Mengting elaborates on the various tasks in Object detection, classification and segmentation. She discusses the following:
			\begin{itemize}
				\item Different approaches in detection, namely CNN's, YOLO and SSD
				\item The different outputs that we expect in the 3 cases of detection, classification and segmentation.
				\item Detailed explanations and motivation of the different CNN architectures, and a comparison between them in object detection.
			\end{itemize}
			\paragraph{Learnings}
			I learned about the various state of the art techniques in object detection. I got to know more about the evolution of CNN architectures, along with their strength and limitations.
			
		\subsection{Ningrong Chen}
			\paragraph{Topics covered}
			Ningrong gives an overview of Object detection and an extensive list of algorithms and neural network architectures alongwith an Introduction to the concept of Model zoo with a lot of resources of pre built zoo models.
			\paragraph{Learnings}
			I learned about Model Zoo - a collection of pre-trained models on different datasets using various algorithms. I also learned about the numerous number of architectures that can be utilized in object detection.
			
	\section{References}
		\begin{itemize}
			\item \href{https://www.youtube.com/watch?v=u6aEYuemt0M\&t=524s}{Andrej Karpathy}
			\item \href{https://en.wikipedia.org/wiki/Computer_vision}{Wikipedia}
			\item \href{https://cs231n.github.io/}{Stanford - cs231n}
			\item \href{http://www.deeplearningbook.org}{Deep Learning Book}
		\end{itemize}
		
	
			
		
\end{document}